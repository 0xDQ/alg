% ----------------------------------------------------------------
% AMS-LaTeX Paper ************************************************
% **** -----------------------------------------------------------
\documentclass{amsart}
\usepackage{graphicx}
\usepackage{amsfonts}
\usepackage{amscd}
\usepackage{amssymb}
\usepackage{mathtools}
% \usepackage{xy}
% ----------------------------------------------------------------
\vfuzz2pt % Don't report over-full v-boxes if over-edge is small
\hfuzz2pt % Don't report over-full h-boxes if over-edge is small
% THEOREMS -------------------------------------------------------
\newtheorem{thm}{Theorem}[section]
\newtheorem{cor}[thm]{Corollary}
\newtheorem{lem}[thm]{Lemma}
\newtheorem{prop}[thm]{Proposition}
\theoremstyle{definition}
\newtheorem{defn}[thm]{Definition}
\newtheorem{alg}[thm]{Algorithm}
\theoremstyle{remark}
\newtheorem{rem}[thm]{Remark}
\numberwithin{equation}{section}
% MATH -----------------------------------------------------------
\newcommand{\Matrix}[4]{ \left( \begin{array}{cc}  #1 & #2 \\  #3 & #4 \\ \end{array} \right) }
\newcommand{\norm}[1]{\left\Vert#1\right\Vert}
\newcommand{\abs}[1]{\left\vert#1\right\vert}
\newcommand{\set}[1]{\left\{#1\right\}}

\newcommand{\NN}{\mathbb N}
\newcommand{\ZZ}{\mathbb Z}
\newcommand{\QQ}{\mathbb Q}
\newcommand{\RR}{\mathbb R}
\newcommand{\CC}{\mathbb C}
\newcommand{\isom}{\cong}
\DeclarePairedDelimiter{\ceil}{\lceil}{\rceil}
\DeclarePairedDelimiter{\floor}{\lfloor}{\rfloor}

\pagestyle{plain}
% ----------------------------------------------------------------
\begin{document}
\title[]{Algorithms Homework 1}%
\author{Evan Simmons \\
        Dept. of Mathematics \& Computer Science \\ University of California Santa Cruz}%
%\date{}
%\dedicatory{}%
%\commby{}%
\renewcommand{\abstractname}{Homework Option}
% ----------------------------------------------------------------
\begin{abstract}
I would like to choose the homework heavy option.
\end{abstract}
\maketitle
% ----------------------------------------------------------------

\section{} Suppose we are given a sorted array of $n$ distinct elements
that has been circularly shifted by $k$ elements. We want to find an
algorithm which runs in $O(log(n))$ time to find the maximum.

\lem \label{lem1}

Consider two elements in our circularly sorted array, $A[i]$ and $A[j]$
where $i < j$. The elements in $A[i..j]$ are sorted if
and only if $A[i] < A[j]$.

\proof{} 

Suppose the array slice $A[i..j]$ is not sorted, then $\exists k \ni
A[k] > A[k+1]$ where $i \leq k \leq j$. We know then that since the
array is circularly sorted that $\forall x \in A[i..k]$ and $\forall y
\in A[(k+1)..j]$ we have $x > y$. Clearly $A[i] \in A[i..k]$ and $A[j] y
\in A[(k+1)..j]$, therefore $A[i] > A[j]$

To show the converse, we simply note that for any sorted array, the
following holds:

$$ \forall i,j \ni i<j \Rightarrow A[i] < A[j] $$

\cor \label{cor1}

Consider two elements in our array $A[i]$ and $A[j]$ where $i<j$, the maximum
is in in $A[i..(j-1)]$ if and only if $A[i] > A[j]$

\proof 

Suppose $A[i] < A[j]$. Then by ~\ref{lem1}, $A[i..j]$ is sorted. Hence
$A[j]$ is larger than every element in $A[i..(j-1)]$, therefore the maximum
cannot be in $A[i..(j-1)]$.

Conversely suppose $A[i] > A[j]$, then by ~\ref{lem1} $A[i..(j-1)]$ is
not sorted. By the definition of a (not completely sorted) circularly
sorted array it has one consecutive pair of elements that are not
pairwise sorted, that pair is the maximum and minimum.

\alg{(cirmax)}

If the array consists of a single element, return that element.

Let $h = \floor{n/2}$; compare the $A[0]$ and $A[h]$ elements. \\

Case (Less than): return the value of cirmax( $A[h..n]$ ) \\

Case (Greater than): return the value of cirmax( $A[0..(h-1)]$ ) \\

\proof

If an array consists of a single element, clearly that element must
be the maximum. If the first element $A[0]$ is less than the middle
element $A[h]$ then by ~\ref{cor1} the maximum occurs in the array slice
$A[h..n]$. However if $A[0]$ is greater than $A[h]$, then, again by
~\ref{cor1} the maximum occurs in $A[0..(h-1)]$.

\prop{The proposed algorithm runs in O(log (n)) time.}

\proof

Since at each level of recursion we do only constant work, and are able
to eliminate half of the array, the complexity of the algorithm is
simply the height of the longest branch of the recursion tree. Recall
that the height of a tree is $log_{b} (n)$ where $b$ branching factor
and $n$ is the total number of elements (including those pruned).
Therefore the given algorithm is $\Theta( log n )$

Alternatively, we could have used the Master Theorem. The recurrence
relation is $T(n) = T(\frac{n}{2}) + n^0$ hence Case 2 applies and
yeilds $T(n) = \Theta (\log n)$
\\


\section{} If $A$ is an array of $n$ numbers. We wish to find:

$$ \max_{1 \leq j<k \leq n} (A[k] - A[j]).$$

\subsection{} Give an $O( n^2 )$ algorithm.
\alg{$O( n^2 )$}

For each element of the array, find its difference with each subsequent element. Return the largest (positive) value of these differences.

\subsection{} Provide a more efficient algorithm.

Can this be done faster than O(n)?

\subsection{} Provide a recurrence relation, and find an upper bound
using the Master Theorem.

\section{} Consider the following sorting algorithm: First sort the
first two thirds of the elements in the array. Next sort the last two
thirds of the array. Finally, sort the first two thirds again.

\subsection{} Provide an informal explaination of why this algorithm returns a
sorted array.



\subsection*{}

\end{document}
% ----------------------------------------------------------------
