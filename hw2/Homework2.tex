% ----------------------------------------------------------------
% AMS-LaTeX Paper ************************************************
% **** -----------------------------------------------------------
\documentclass{amsart}
\usepackage{graphicx}
\usepackage{amsfonts}
\usepackage{amscd}
\usepackage{amssymb}
\usepackage{amsthm}
\usepackage{mathtools}
\usepackage{inconsolata}
\usepackage[shortlabels]{enumitem}
% \usepackage{xy}
% ----------------------------------------------------------------
\vfuzz2pt % Don't report over-full v-boxes if over-edge is small
\hfuzz2pt % Don't report over-full h-boxes if over-edge is small
% THEOREMS -------------------------------------------------------
\newtheorem{thm}{Theorem}[section]
\newtheorem{cor}[thm]{Corollary}
\newtheorem{lem}[thm]{Lemma}
\newtheorem{prop}[thm]{Proposition}
\newtheorem{claim}[thm]{Claim}
\newtheorem*{claimstar}{Claim}
\theoremstyle{definition}
\newtheorem{defn}[thm]{Definition}
\newtheorem{alg}[thm]{Algorithm}
\newtheorem*{algstar}{Algorithm}
\theoremstyle{remark}
\newtheorem{rmk}[thm]{Remark}
\newtheorem*{rmkstar}{Remark}
\numberwithin{equation}{section}
% MATH -----------------------------------------------------------
\newcommand{\Matrix}[4]{ \left( \begin{array}{cc}  #1 & #2 \\  #3 & #4 \\ \end{array} \right) }
\newcommand{\norm}[1]{\left\Vert#1\right\Vert}
\newcommand{\abs}[1]{\left\vert#1\right\vert}
\newcommand{\set}[1]{\left\{#1\right\}}

\newcommand{\NN}{\mathbb N}
\newcommand{\ZZ}{\mathbb Z}
\newcommand{\QQ}{\mathbb Q}
\newcommand{\RR}{\mathbb R}
\newcommand{\CC}{\mathbb C}
\newcommand{\isom}{\cong}
\DeclarePairedDelimiter{\ceil}{\lceil}{\rceil}
\DeclarePairedDelimiter{\floor}{\lfloor}{\rfloor}

\let\null\varnothing

\pagestyle{plain}
% ----------------------------------------------------------------
\begin{document}
\title[]{Algorithms Homework 1}%
\author{Evan Simmons \\
        Dept. of Mathematics \& Computer Science \\ University of California Santa Cruz}%
%\date{}
%\dedicatory{}%
%\commby{}%
\renewcommand{\abstractname}{Homework Option}
% ----------------------------------------------------------------
\begin{abstract}
I have chosen the homework heavy option.
\end{abstract}
\maketitle
% ----------------------------------------------------------------

\section{} Let $G = G(V, E)$ be an arbitrary, connected, undirected graph with
vertex set $V$ and edge set $E$, where each $e \in E$ has a distinct
positive length le. Let $f : \RR^+ \rightarrow \RR^+$ be strictly
increasing. We want to compute a subset of edges $S \subseteq E$ such
that:

\begin{enumerate}[(a)]
  \item $G(V,S)$ is connected.
  \item $\sum_{e \in S} f(l_e)$ is as small as possible.
\end{enumerate}

What is the smallest number of evaluations of $f$ needed to do this?

\rmkstar We consider the output of this function for each edge length just
as we would the edge weight in the minimum spanning tree problem since
it is exactly the quantity we seek to minimize.

\claimstar We must evaluate the function $f$ at \textit{least} once for every
edge in a cycle.

\proof Consider an arbitrary edge $e$ in our graph.

If it is not in a cycle, removing it would render the graph
disconnected, therefore we need not consider $f(e)$.

If it is in a cycle, assume towards a contradiction that there exists
an optimal solution $G(V,S)$ that does not evaluate $f(e)$. Since we
are free to choose any one edge in a cycle to remove and still preserve
connectetness, if $f(e)$ happens to be larger than the edge removed in
$G(V,S)$, then clearly the solution is not in fact optimal.

\section{} Give an algorithm that given a sequence of $n$ positive real numbers $W = w_1, w_2,
\ldots, w_n$ and a real number $B$ partitions $W$ into contiguous subsequences, i.e.,

$$ S = \{ (w_1,w_2,\ldots,w_k), (w_{k+1},w_{k+2},\ldots,w_l),\ldots,(w_q,w_{q+1},\ldots,w_n) \} $$

such that the sum of the numbers in each subsequence is at most $B$ while
minimizing the number of subsequences. (Assume $w_i \leq B$.)

\algstar We iterate through the sequence of $w$'s in order. If our set
of subseqences is empty, we create a new subsequence and append the
current element. Otherwise, for each $w_i$ we attempt to add it to
the last subsequence in $S$, if this fails by making the sum of the
subseqence more than $B$, we instead create a new subsequence and append
$w_i$.

\proof If $W = \null$ then we have a set with an empty sequence, and
are done. Assume towards a contradiction that for some $W$ we have an
optimal solution $O$ which differs from the solution produced by the
above algorithm $S$. Let $w_j$ denote the element of first difference,
which corresponds to the subsequence that $w_j$ is placed in. There are
two cases, in the first:

\begin{align*}
  S &= \{ \ldots, \overbrace{\{w_k, \ldots  w_{j-1}\} }^{S_i}, \{w_j, w_{j+1},\ldots \}, \ldots \} \\
  O &= \{ \ldots, \underbrace{\{w_k, \ldots w_{j-1}, w_j \}}_{O_i}, \{w_{j+1},\ldots \}, \ldots \}.
\end{align*}

But we immediately see that the sum of $O_i > B$ by the definition of our
algorithm.$\Rightarrow \Leftarrow$

In the second case:

\begin{align*}
  S &= \{ \ldots, \overbrace{\{w_k, \ldots  w_{j-1}, w_j \} }^{S_i}, \{w_{j+1},\ldots \}, \ldots \} \\
  O &= \{ \ldots, \underbrace{\{w_k, \ldots w_{j-1} \}}_{O_i}, \{ w_j, w_{j+1},\ldots \}, \ldots \}.
\end{align*}

But then we have shown that for the first $j$ elements, our algorithm does better than
that of optimal. $\Rightarrow \Leftarrow$

Hence the output of our algorithm is the unique optimal solution.

\rmkstar Clearly our algorithm runs in $O(n)$ time since we traverse the
set $W$ once.

\section{}

We wish to manufacture n distinct hardware items. Each item needs to go
through 3 stages of processing. The first, and most important stage,
called \textit{design} can only be performed by our master designer who works
by starting work on an item, designing it through to the end, and only
then starting on another item. The remaining two phases, called \textit{assembly}
and \textit{testing}, are outsourced and for each of them there is an infinite
supply of people who can perform the corresponding task as soon as it is
assigned to them. Naturally, each item first needs to be designed, then
assembled, and then tested.

Each item $i$ requires $d_i$ hours of design, $a_i$ hours of assembly, and $t_i$
hours of testing. We are interested in determining the order in which we
should design the items so that we minimize the time by which all items
will be ready.

Give a $O(n\ log\ n)$ algorithm which takes as input $n$ triples
$(d_i,a_i,t_i)$ and determines the optimal design order.

\claimstar It is sufficient to start the jobs in order of the sum of
their respective assembly and testing times.

XXX: TODO
\proof Assume towards a contradiction that there exists an optimal
schedule $\sigma$ whose latest finishing job ends before the
aforementioned schedule $\tau$. Let $l (s)$ denote the latest
finishing job of any schedule $s$.

\alg We employ mergesort wiht the comparator function that compares
the sum of the assembly and testing times of each element.

\section{} Let $G = G(V, E)$ be a directed graph representing a
road network. That is, an arc $(u, v)$ from vertex $u$ to vertex
$v$ corresponds to a road connecting location $u$ with location $v$
directly. Going beyond the standard shortest-path setting, we will now
also take into account traffic by accepting that the amount of time it
takes to cross an arc $e = (u, v)$ depends on the time that one starts
of at $u$. Specifically, we will assume that based on past observations
of traffic patterns we have for each arc $e = (u, v)$ a function $g_e$
which takes as input the time one leaves $u$ and returns the time that
they will arrive at $v$. The functions are such that for every arc $e$,
we have $g_e(t) \geq t$ for every time $t$ and $g_e(t_1) < g_e(t_2)$,
whenever $t_1 < t_2$.

Modify Dijkstra’s algorithm to get a $O(|E|\cdot log\ |E|)$ algorithm which
finds the soonest that one can reach a given vertex $t$ of $G$ if they
leave a given vertex $s$ of $G$ at a given time $t_0$.

\claimstar We can simply use Dijkstra's algorithm while maintaining a time
at which we reached each node to pass to the $g$ function when we
explore its edges.

XXX: TODO
\algstar Not sure about this.

\proof Of correctness

\proof Of time


\section{} Let $G = G(V,E)$ be an arbitrary, connected, undirected graph
with vertex set $V$ and edge set $E$. Assume that each $e \in E$ has a
distinct positive length length. Define \textit{tediousness} of any path
$P$ in $G$ to be the length of the longest edge in tediousness of any
path $P$ in $G$ to be the length of the longest edge in $P$. Define the
\textit{tediousness} between vertices $u$ and $v$ to be the tediousness
of the least tedious path connecting $u$ and $v$ in $G$.

\textbf{Fun Subgraph Problem (FSP).} Find a subgraph of edges $S \subseteq E$ such that:

\begin{enumerate}[(a)]
  \item $G(V,S)$ is connected.
  \item $\sum_{e \in S} l_e$ is as small as possible.
  \item For every pair of vertices $u,v \in V$, the \textit{tediousness} between $u,v$ in $G(V,S)$ 
    is no greater than the \textit{tediousness} between $u,v$ in $G(V,E)$.
\end{enumerate}

Prove or disprove the following:

XXX: TODO
\subsection{}

\subsection{}

\subsection{}

\end{document}
